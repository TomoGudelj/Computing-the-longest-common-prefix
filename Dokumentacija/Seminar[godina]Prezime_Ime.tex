\documentclass[times, utf8, seminar, numeric]{fer}
\usepackage{array}
\newcolumntype{P}[1]{>{\centering\arraybackslash}p{#1}}
\usepackage{booktabs}
 \usepackage{url}

\begin{document}

% Ukljuci literaturu u seminar
\nocite{*}

% TODO: Navedite naslov rada.
\title{Računanje najduljeg zajedničkog prefiksa temeljeno na BWT}

% TODO: Navedite vaše ime i prezime.
\author{Silvestar Badak, Tomislav Gudelj, Domagoj Vukadin}

% TODO: Navedite ime i prezime voditelja.
\voditelj{doc. dr. sc. Mirjana Domazet-Lošo}

\maketitle

\tableofcontents

\chapter{Uvod}
Uvod.

\chapter{Strukture podataka}

\section {Sufiksno polje}

\section {Burrows-Wheelerova transformacija - BWT}

\section {Binarno stablo valića}

\chapter{Algoritam 1}

\chapter{Algoritam 2}

\chapter{Eksperimenti i rezultati}

	  \begin{table}[h]
	  	
	  	\centering
	  	\begin{tabular}{|P{2cm}|P{3cm}|P{2.5cm}|} 
		 \hline
		  Broj znakova & Implementacija Simona Goga [s] & Binarno stablo valića [s] \\
	  		 \hline
	  		100 & 0.037231 & x \\
	  		1,000 & 0.030089 & x \\
	  		10,000 & 0.022991  & x \\ 
	  		100,000 &  0.03929 & x \\
	  		1,000,000 & 0.223351 & x \\	
	  		
	  		\hline  		
	  	\end{tabular}
	  	\caption{Brzine izvođenja}
	  	\label{tbl:std_dev}
	  	
	  \end{table}




\chapter{Zaključak}
Zaključak.

\bibliography{literatura}
\bibliographystyle{fer}

\chapter{Sažetak}
Sažetak.

\end{document}
